\subsection{Interference laws and Young's double slit experiment}
    I consider here simple geometry of the double slit experiment and show the condition under which one can disintegrate transverse coherence properties of the radiation. 
    
    The field can be written as the sum of two fields 
    \begin{align}
        E(\vec{r}, t) = A_1 E(\vec{r_1}, t + \Delta t) + A_2 E(\vec{r_2}, t - \Delta t), 
    \end{align}
    where $t_1 = R_1/c$ and $t_2 = R_2/c$. The observed intensity of this radiation is:
    \begin{align}
        I(\vec{r}, t) = |A_1|^2 I(\vec{r}_1, t + \Delta t) + |A_2|^2 I(\vec{r}_2, t - \Delta t) + 
        2 \mathfrak{Re} \{ A_1^* A_2 E^*(\vec{r_1}, t + \Delta t) E(\vec{r_2}, t - \Delta t) \}
    \end{align}
    And rewriting is the the statistical average oven an ensemble and knowing that $\langle I(\vec{r}_{1,2}, t \pm \Delta t) \rangle = I(\vec{r}_{1,2})f(\bar{t} \pm \Delta t)$:
    \begin{align}
        \langle I(\vec{r}, t) \rangle = 
        |A_1|^2 I(\vec{r}_{1})f(\bar{t} + \Delta t) + 
        |A_2|^2 I(\vec{r}_{2})f(\bar{t} - \Delta t) + 
        2 \mathfrak{Re} \{ A_1^* A_2 \langle E^*(\vec{r_1}, t + \Delta t) E(\vec{r_2}, t - \Delta t) \rangle\}
    \end{align}
    as one can see $\langle E^*(\vec{r_1}, t - t_1) E(\vec{r_2}, t - t_2) \rangle = \Gamma (\vec{r_1}, \vec{r_2}, t_1, t_2) = g_t(\vec{r}_1, \vec{r}_2, \Delta t) f(\bar{t})$. 
    And now let's assume that our radiation is quasi-stationary and use Eq.~\ref{Eq:weak_quasi_homo} and write:
    \begin{align}
        \langle I(\vec{r}, t) \rangle = 
        |A_1|^2 I(\vec{r}_{1})f(\bar{t}) + 
        |A_2|^2 I(\vec{r}_{2})f(\bar{t}) + 2|A_1| |A_2| f(\bar{t}) \mathfrak{Re} \big\{ g_t(\vec{r}_1, \vec{r}_2, \Delta t)\big\},
    \end{align}
    here I assumed that the time difference $\Delta t$ is much smaller than the radiation coherence time: $\Delta t \ll \tau_c < \sigma_{\tau}$, where $ \sigma_{\tau}$ is the pulse duration determined by the distribution $f(\bar{t})$. Interesting to note that the expression is the "instant" interference, as written for specific time, to obtain the real observed signal one need to average over time response function of a detector.

    $g_t$ function can be expressed as the following:
    \begin{align}
        g_t(\vec{r}_1, \vec{r}_2, \Delta t) = |g(\vec{r}_1, \vec{r}_2, \Delta t)|e^{i \{\alpha(\vec{r}_1, \vec{r}_2, \Delta t) - \omega \Delta t\}}, 
    \end{align}
    where $\alpha(\vec{r}_1, \vec{r}_2, \Delta t) = \textup{arg} g_t(\vec{r}_1, \vec{r}_2, \Delta t) + \omega \Delta t$. With chosen $\Delta t$ I notice that $|g(\vec{r}_1, \vec{r}_2, t_2 - t_1)|$ and $Arg(g(\vec{r}_1, \vec{r}_2, t_2 - t_1))$ varies very slowly with respect to this time difference and I can approximate $g_t$:
    \begin{align}
        g_t(\vec{r}_1, \vec{r}_2, \Delta t) \approx g_t(\vec{r}_1, \vec{r}_2, 0) e^{-i \omega \Delta t}
    \end{align}
    $g_t(\vec{r}_1, \vec{r}_2, 0) = J(\vec{r}_1, \vec{r}_2)$ is simply called mutual intensity function.
    and I notice the relation 
    \begin{align}
        \langle E(\vec{r}_1, \bar{\omega})E^*(\vec{r}_1, \bar{\omega})\rangle = G_{\omega}(\vec{r}_1, \vec{r}_2, \bar{\omega}) = \frac{1}{2 \pi} \int \limits_{-\infty}^{\infty}  g_t(\vec{r}_1, \vec{r}_2, \Delta t) e^{i \bar{\omega} \Delta t} d \Delta t
    \end{align} 